\Chapter{Voorwoord}

\begin{OnehalfSpace}
\emph{God, onze Schepper, ik dank U voor Uw grenzeloze mededogen en liefde; ik prijs Uw schepping, want er is geen minuut van mijn leven dat ik niet geniet van Uw wonderbaarlijke werken: van de zuurstof die ik adem tot mijn ouders, vrienden, familie en dochter van wie ik ziels veel houd; de zwaartekracht op deze prachtige planeet en het voedzame eten dat er op groeit; mijn lichaam en mijn geest -- voor dit alles dank ik U, lieve God.}
\end{OnehalfSpace}
\vspace{1cm}

\noindent Dit boek bundelt teksten die vrij verkrijgbaar zijn op de volgende websites:
\begin{itemize}
  \item \href{http://quranix.com}{\textbf{quranix.com}} -- het Arabische origineel een de Engelse vertaling door Edip Yüksel, Laith Saleh al{-}Shaiban \& Martha Schulte{-}Nafeh (hier na de Yüksel vertaling genoemd)
  \item \href{http://bijbelenkoran.nl}{\textbf{bijbelenkoran.nl}} -- de vertaling van Fred Leemhuis
  \item \href{http://al-quran.info}{\textbf{al-quran.info}} -- de vertaling door Sofian S. Siregar
\end{itemize}
Het doel van deze bundel: Koran studie laagdrempeliger maken voor Nederlandstaligen.
\vspace{0.3cm}

\noindent Ondanks dat veel mensen van de Koran's bestaan afweten zijn er maar weinigen hem lezen.  Soms wordt de Koran zo heilig geacht dat men misinterpretatie vreest, waarna men de uitleg van een derde (vaak een Imam of een zogenaamde ``Islam geleerde'') verkiest boven directe studie de Koran.  Ookwel wordt het correct uitspreken en onthouden van de Arabische text belangrijker geacht dan begrip en implementatie van de boodschap.

Hoe minder de Koran wordt gelezen, hoe minder men \emph{uit eigen ervaring} weet wat er in staat, met als treurig gevolg dat men zich bazeert op informatie van horen-zeggen.  Dit terwijl in de Koran God ons juist opdraagt de Koran (op) te lezen (\ql{2:121}, \ql{35:29}), en te luisteren als deze opgelezen wordt (\ql{7:204}, \ql{39:18}).

Een andere belemmering voor Koran studie is dat informatie uit de Koran vaak wordt vermengd met informatie uit andere bronnen.  Bijvoorbeeld informatie uit de hadith-collecties die pas meer dan 200 jaar na het overlijden van de profeet zijn vastgelegd.  Voor het overgrote deel van de islamieten zijn hadith-collecties van essentieel belang voor \emph{hun} religie.  Helaas heeft dit tot gevolg dat de meeste vertalingen van de Koran naast het Arabische origineel ook gebazeert zijn de hadith-collecties.  Hierdoor krijgen lezers van deze vertalingen een vertekend beeld van Gods woord.

In de Koran wordt dit al aangekondigd.  Er staat er dat men zal proberen het bericht van de Koran te verdraaien (\ql{41:26}, \ql{48:15}) maar dat het niet mogelijk is omdat God het boek heeft beschermd (\ql[56:77-56:81]{56:77}).
\vspace{0.3cm}

\newpage
\mbox{~}
\vspace{3cm}

\noindent In dit boek staan slechts drie vertalingen, meer vertalingen naast elkaar zou simpelweg niet praktisch zijn.  De vertalingen van Siregar en Leemhuis worden vaak genoemd als beste Nederlandse vertalingen voor Koran studie doeleinden.  Door meerdere vertalingen naast elkaar te tonen kan gemakkelijk vergeleken worden om zo tot een beter begrip te komen van de originele betekenis.

De vertaling van Siregar is ingescand en omgezet tot text, hetgeen veel foutjes heeft geïntroduceerd.  Deze foutjes zijn makkelijk te negeren; bijvoorbeeld een letter ``c'' die eigenlijk een ``e'' had moeten zijn zal niet ineens tot een misinterpretatie leiden.

Aangezien de meeste Nederlanders het Engels goed beheersen is voor één van de drie vertalingen een Engelse gekozen: de Yüksel vertaling.  Deze vertaling volgt nauwlettend het Arabische origineel; het is een letterlijke, bijna woord-voor-woord vertaling.  De auteurs van de Yüksel vertaling hebben expliciet geen gebruik gemaakt van de hadith-collecties of anderzijds dogmatische interpretaties.  Helaas bestaat er nog geen Nederlandse vertaling die gelijke uitgangspunten heeft, vandaar dat deze Engelse vertaling onderdeel is van deze bundel.

De Yüksel vertaling bevat sterretjes op de plekken waar in de originele versie naar voetnoten wordt verwezen, op \href{http://quranix.com}{quranix.com} staan alle voetnoten ter referentie.
\vspace{0.3cm}

\noindent In geval van twijfel over de betekenis van een bepaald vers in de Koran is het raadzaam om één of meer woord-voor-woord vertalingen te raadplegen.  Op de volgende websites vinden we zulke vertalingen:
\begin{itemize}
  \item \href{http://allahsquran.com/learn}{allahsquran.com}\footnote{\textsf{http://allahsquran.com/learn}} -- bied mogelijkheid om vertaling toe te voegen
  \item \href{http://corpus.quran.com/wordbyword.jsp}{corpus.quran.com}\footnote{\textsf{http://corpus.quran.com/wordbyword.jsp}} -- toont Arabische grammaticale zinsontleding
  \item \href{http://www.islamicstudies.info/wordtranslation.php}{islamicstudies.info}\footnote{\textsf{http://www.islamicstudies.info/wordtranslation.php}} -- verzameling plaatjes
  \item \href{http://www.quraninenglish.com/cgi-local/pages.pl?/contents}{quraninenglish.com}\footnote{\textsf{http://www.quraninenglish.com/cgi-local/pages.pl?/contents}} -- verzameling gescande plaatjes
\end{itemize}
Als een woord verwarring veroorzaakt is het raadzaam te onderzoeken hoe het woord op andere plekken in de Koran gebruikt wordt.  Hierbij kan onderscheid gemaakt worden tussen woorden die binnen de Koran één-enkele betekenis hebben, en woorden met meerdere betekenissen.  Voor de woorden met meerdere betekenissen is het vrijwel altijd mogelijk om aan de hand van de context op te maken welke betekenis geaccepteerd dient te worden.
\vspace{0.3cm}

\noindent Dit boek is vervaardigd met een behulp van een computer programma dat door de \emph{Open Quran Collective} werkgroep is gemaakt.  Op de \href{http://github.com/oqc/study-quran}{GitHub pagina}\footnote{\textsf{http://github.com/oqc/study-quran}} van dit project wordt de broncode van dit programma vrij beschikbaar gesteld.  Met opmerkingen kunt u terecht op de \href{http://github.com/oqc/study-quran/issues}{issue tracker}\footnote{Ga naar \textsf{http://github.com/oqc/study-quran/issues} en klik op de ``New Issue'' knop.}.  De \emph{Open Quran Collective} werkgroep verwelkomt iedere vorm van hulp.




